\chapter{Nachweis der hermiteschen Form von \emph{H}}
Eine hermitesche Matrix oder selbstadjungierte Matrix wird im mathematischen Teilgebiet der Linearen
Algebra untersucht. Es handelt sich um eine spezielle Art von quadratischen Matrizen. Benannt sind
diese nach dem Mathematiker Charles Hermite\footnote{Charles Hermite (* 24. Dezember 1822 in Dieuze
(Lothringen); † 14. Januar 1901 in Paris) war ein französischer Mathematiker.}.

\section{Definition - hermitesch}
Eine \(n \times n\)-Matrix \(A\) mit Einträgen in \(\mathrm{C}\) heißt hermitesch, wenn sie mit
ihrer (hermitesch) Adjungierten \(A^*\), also der transponierten und komplex konjugierten Matrix
übereinstimmt. Das heißt, wenn
\begin{equation}
    A = A^* = \overline A^{\mathrm T} = \overline{A^{\mathrm T}}
\end{equation}
gilt.

Für die adjungierte Matrix finden sich auch die Bezeichnungen \(A^\dagger\) und \(A^{\mathrm H}\).
Die Schreibweise \(A^*\) wird auch für die komplex konjugierte Matrix verwendet.

Für die Einträge einer hermiteschen Matrix gilt also:
\begin{equation}
\label{eq:elements}
    a_{jk} = \overline{a_{kj}}
\end{equation}
Anders formuliert ist eine Matrix A genau dann hermitesch, wenn ihre Transponierte gleich ihrer
komplex Konjugierten ist, d.h. \(A^{\mathrm T} = \overline A\).

\subsection{direkte Folgen aus der Definition}
\begin{itemize}
  \label{item:hermSym}
  \item Der Realteil einer hermiteschen Matrix ist symmetrisch, \(\mathrm{Re}(a_{jk}) =
        \mathrm{Re}(a_{kj})\), der Imaginärteil ist schiefsymmetrisch, \(\mathrm{Im}(a_{jk}) =
        -\mathrm{Im}(a_{kj})\).
  \item Ist \(A\) hermitesch, dann ist \(iA\) schiefhermitesch.
  \item Die Hauptdiagonalelemente einer hermiteschen Matrix sind reell.  
  \item Für reelle Matrizen fallen die Begriffe hermitesch und symmetrisch zusammen.
  \item Hermitesche Matrizen sind normal, d. h \(A^* \cdot A = A\cdot A^*\)
\end{itemize}

\section{Mathematischer Beweis für die gegebene Matrize 'H'}
Mit der gegebenen Bildungsvorschrifft
\begin{equation}
    H := \left(\frac{1}{j+k+1}\right)_{j,k=0}^{n}
\end{equation}
Gleichung \ref{eq:elements} und den Aussagen aus \ref{item:hermSym} folgt, dass für \(H\) der
folgende Zusammenhang gillt:
\begin{align}
                                       H &= H^{\mathrm T} &&| \text{ da rein reelwertig}      \\
                                 H_{j,k} &= H_{k,j} &&| \text{ Indizees vertauschbar}         \\
\left(\frac{1}{j+k+1}\right)_{j,k=0}^{n} &= \left(\frac{1}{j+k+1}\right)_{k,j=0}^{n} &&
\end{align}
somit ist \(H\) symmetrisch und auch hermitesch.