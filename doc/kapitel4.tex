\chapter{Numerische Lösung des LGS für eine fünfstellige Genauigkeit}
Falls H eine \(n \times n\)-Matrix ist, wird von Octave Gaußelimination \footnote{Johann Carl
Friedrich Gauß (latinisiert Carolus Fridericus Gauss; * 30. April 1777 in Braunschweig; † 23.
Februar 1855 in Göttingen) war ein deutscher Mathematiker, Astronom, Geodät und Physiker mit einem
breit gefächerten Feld an Interessen.} verwendet, sonst wird via QR-Zerlegung eine Lösung im Sinne
der kleinsten Fehlerquadrate berechnet. Ist A schlecht konditioniert oder singulär, wird eine
Warnung ausgegeben. Um zu zeigen, dass das Eliminationsverfahren nach Gauß ungeeignet ist haben wir
hierführ eine eigene Funktion gechrieben, welche im Listing \ref{code:gauss} zu sehen ist.

\lstset{label=code:gauss, caption=Gauß-Eliminationsverfahren als Matlab-/Octavescript}
\lstinputlisting{../src/gaussElim.m}