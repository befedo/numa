\chapter{Eigenwerte und Kondition von H}
Die Erstellung der Matrix \(H\) wird durch ein Octave-Script\footnote{GNU Octave ist eine freie
Software zur numerischen Lösung mathematischer Probleme, wie zum Beispiel Matrizenrechnung, Lösen
von (Differential-)Gleichungssystemen, Integration etc. Berechnungen können in Octave mit einer
Skriptsprache durchgeführt werden, die weitgehend zu dem proprietären MATLAB kompatibel ist.}
organisiert. Siehe hierzu \ref{code:foo} und \ref{code:numa}.

Die berechneten Daten werden in zwei Dateien geschrieben,
\lstset{language=Octave, inputencoding=utf8, literate={Ö}{{\"O}}1 {Ä}{{\"A}}1 {Ü}{{\"U}}1
{ß}{{\ss}}2 {ü}{{\"u}}1 {ä}{{\"a}}1 {ö}{{\"o}}1}
\begin{lstlisting}
    # öffnen und schreiben in eine Datei
    # mögliche Modi: r-read, w-write, a-append
    fileId = fopen ('fileName','mode');
    # do something
        ...
    # done!
    # schließen der Datei nicht vergessen
    fclose (fileId);
\end{lstlisting}
um sie später mit einem externen Programm (z.B. gnuplot) weiter verarbeiten zu können. Ihnen sind
die Werte in Tabelle \ref{tab:task3} entnommen.

\vspace{1em}
\centering    
\begin{table}[htbp]
\tiny
\caption{Eigenwerte \& Konditionen für 3..10xn Matrizen \label{tab:task3}}         
\begin{tabularx}{\textwidth}{|c|X|X|X|X|X|X|X|X|X|X|X|}
    \hline    
    N &&&&&&&&&&& Kond.\\\hline
    3 & 0.00268 & 0.12232 & 1.40831 &&&&&&&& 524.06\\\hline
    4 & 9.6702 e-05 & 6.7383 e-03 & 1.6914 e-01 & 1.5002 e+00 &&&&&&& 1.5514 e+04\\\hline
    5 & 3.2879 e-06 & 3.0590 e-04 & 1.1407 e-02 & 2.0853 e-01 & 1.5671 e+00 &&&&&& 4.7661 e+05\\\hline
    6 & 1.0828 e-07 & 1.2571 e-05 & 6.1575 e-04 & 1.6322 e-02 & 2.4236 e-01 & 1.6189 e+00 &&&&& 1.4951 e+07\\\hline
    7 & 3.4939 e-09 & 4.8568 e-07 & 2.9386 e-05 & 1.0086 e-03 & 2.1290 e-02 & 2.7192 e-01 & 1.6609 e+00 &&&& 4.7537 e+08\\\hline
    8 & 1.1115 e-10 & 1.7989 e-08 & 1.2943 e-06 & 5.4369 e-05 & 1.4677 e-03 & 2.6213 e-02 & 2.9813 e-01 & 1.6959 e+00 &&& 1.5258 e+10\\\hline
    9 & 3.4998 e-12 & 6.4609 e-10 & 5.3856 e-08 & 2.6730 e-06 & 8.7581 e-05 & 1.9789 e-03 & 3.1039 e-02 & 3.2163 e-01 & 1.7259 e+00 && 4.9315 e+11\\\hline
    10& 1.0931 e-13 & 2.2667 e-11 & 2.1474 e-09 & 1.2290 e-07 & 4.7297 e-06 & 1.2875 e-04 & 2.5309 e-03 & 3.5742 e-02 & 3.4293 e-01 & 1.7519 e+00 & 1.6025 e+13\\\hline
\end{tabularx}
\end{table}