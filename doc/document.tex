\input{header}

\begin {document}
	%% Titelseite
	\pdfbookmark[1]{Titelseite}{titlepage}
	\linespread{1.05}\selectfont % 1,5-zeilig aus

\titlehead{
	\centering
	\includegraphics[width=\textwidth]{img/unilogo.png}\\
	\vspace{1em}
}

\subject{Beleg}
\title{Numerische Mathematik - Problem2: schlecht konditioniertes LGS}
\subtitle{\vspace{2cm}}

\author{\textbf{\parbox{0.9\textwidth}{
    \centering
    Marc Ludwig\\
    Marvin Hohlfeld\\
    Matthias Bukovsky\\
    Matthias Springstein}}}
\date{\normalsize Jena, \today}

\publishers{\large 

\begin{tabular}{rl}
Betreuer:& Prof. P. Wilde
\end{tabular}
}

\maketitle

	%% Inhaltsverzeichnis
	\newpage
	\thispagestyle{empty}
	\null

	\pagenumbering{Roman}
	\clearpage\setcounter{page}{1}

	\pdfbookmark[1]{Inhaltsverzeichnis}{tableofcontents}
	\tableofcontents
	\pagebreak
	
	\newpage
	\thispagestyle{empty}
	\null	

    \renewcommand{\abstractname}{Abstract}    
    \begin{abstract}
      Begleitend zur Lehrveranstaltung "`Numerische Mathematik"' im Sommersemester 2013, soll Anhand
      eines schlecht konditionierten Gleichungssystems aufgezeigt werden wie sich minimale Störungen
      der "`rechten Seite"' auf die Lösung des Systems auswirken.
      Dieses Projekt wurde zudem mit einem Versionskontrollsystem bearbeitet und steht ausdrücklich
      zur Weiterbearbeitung zur Verfügung, unter \url{https://github.com/befedo/numa}.
    \end{abstract}

    %% Eigentlicher Inhalt
    \pagenumbering{arabic}
    \clearpage\setcounter{page}{1}

	\chapter{Entwickle die Funktion \emph{f} aus obigem Problem in eine Taylorreihe}

\section{Die Taylorreihe}
In der Analysis verwendet man Taylorreihen (auch Taylor-Entwicklungen, nach dem Mathematiker Brook
Taylor\footnote{Brook Taylor (* 18. August 1685 in Edmonton, Middlesex; † 29. Dezember 1731 in
Somerset House, London) war ein britischer Mathematiker. Nach ihm wurde u.a. die Taylorreihe
benannt.}), um Funktionen in der Umgebung bestimmter Punkte durch Potenzreihen darzustellen
(Reihenentwicklung). So kann ein komplizierter analytischer Ausdruck durch eine nach wenigen
Gliedern abgebrochene Taylorreihe (oftmals gut) angenähert werden, z.B. in der Physik oder bei der
Ausgleichung geodätischer Netze: So ist die oft verwendete Kleinwinkelnäherung des Sinus eine nach
dem ersten Glied abgebrochene Taylorreihe dieser Funktion.

\section{Definition}
Sei \( I \subset \mathrm{R} \) ein offenes Intervall, \( f \colon I \rightarrow \mathrm{R} \) eine
unendlich oft differenzierbare Funktion und \(a\) ein Element von \(I\). Dann heißt die unendliche
Reihe
\begin{align}
        \label{eq:taylorreihe1}
P_f(x)  & = f(a) + \frac{f'(a)}{1!} (x-a) + \frac{f''(a)}{2!} (x-a)^2 + \dotsb +
            \frac{f^{(n)}(a)}{n!} (x-a)^n + \dotsb\\
        \label{eq:taylorreihe2}
        & = \sum_{n=0}^\infty \frac{f^{(n)}(a)}{n!} (x-a)^n
\end{align}
die Taylor-Reihe von f mit Entwicklungsstelle a.

Hierbei bezeichnet
\begin{itemize}
    \item   \(n!\) die Fakultät \(n! = 1 \cdot 2 \cdot \dots \cdot n\) und
    \item   \(f^{(n)}(a)\) die n-te Ableitung von \(f\) an der Stelle a, wobei man \(f^{(0)} := f\)
            setzt.
\end{itemize}

\section{Eigenschaften}
Die Taylorreihe \(P_f(x)\) zur Funktion \(f(x)\) ist eine Potenzreihe \(P\) in \(x\). Im Fall einer
analytischen Funktion \(f\) hat die Taylor-Reihe in jedem Punkt \(a\) einen positiven
Konvergenzradius \(r\) und stimmt in ihrem Konvergenzbereich mit \(f\) überein, d.h. es gibt ein
\(r>0\), sodass für \(|x-a|<r\) gilt
\begin{equation}
    P_f(x) = \sum_{n=0}^\infty a_n (x-a)^n = f(x)
\end{equation}.
Außerdem stimmen die Ableitungen der Reihe im Entwicklungspunkt \(a\) mit den tatsächlichen
Ableitungswerten überein:
\begin{equation}
    P_f^{(k)}(a) = f^{(k)}(a)
\end{equation}
.
    
\section{Konstruktion}
Damit die Ableitungen der Reihe im Entwicklungspunkt \(a\) mit den tatsächlichen Ableitungswerten
übereinstimmen, sind die Koeffizienten \(a_n\) der Taylorreihe wie folgt konstruiert:
\begin{equation}
    \label{eq:konstruktion}
    a_n = \frac{f^{(n)}(a)}{n!} \Rightarrow P_f(x) = \sum_{n=0}^\infty \frac{f^{(n)}(a)}{n!} (x-a)^n
\end{equation}
.

\section{Entwicklung von \emph{f(x)} in eine Tylor Reihe}
Bei gegebener Funktion
\begin{equation}
    f\left(x\right) = \frac{1}{1+x}
\end{equation}
und den Gleichungen \ref{eq:taylorreihe1} und \ref{eq:taylorreihe2} folgt für deren Ableitungen
\begin{align}
    f'\left(x\right)    &= -\frac{1}{\left(1+x\right)^2} \\
    f''\left(x\right)   &=  \frac{2\left(1+x\right)}{\left(1+x\right)^4}
                         =  \frac{2}{\left(1+x\right)^3} \\
    f'''\left(x\right)  &=  \frac{2\left(1+x\right)^4 -8\left(1+x\right)\left(1+x\right)^3}
                            {\left(1+x\right)^8}
                         =  -\frac{6}{\left(1+x\right)^4}
\end{align}
Hieraus läßt sich folgende Bildungsvorschrifft für eine beliebige Ableitung bestimmen
\begin{equation}
    f^n\left(x\right) = \left(-1\right)^n \frac{n!}{\left(1+x\right)^{n+1}}
\end{equation}
hieraus folgt nach Gleichung \ref{eq:konstruktion}
\begin{align}
    f\left(x\right)  =  P_f\left(x\right) 
                    &= \sum_{n=0}^{\infty} \left(-1\right)^n
                        \frac{n!}{n!\left(1+a\right)^{n+1}}\left(x-a\right)^n\\
                    &=  \sum_{n=0}^{\infty} \left(-1\right)^n
                        \frac{\left(x-a\right)^n}{\left(1+a\right)^{n+1}}
\end{align}
	\chapter{Nachweis der hermiteschen Form von \emph{H}}
Eine hermitesche Matrix oder selbstadjungierte Matrix wird im mathematischen Teilgebiet der Linearen
Algebra untersucht. Es handelt sich um eine spezielle Art von quadratischen Matrizen. Benannt sind
diese nach dem Mathematiker Charles Hermite\footnote{Charles Hermite (* 24. Dezember 1822 in Dieuze
(Lothringen); † 14. Januar 1901 in Paris) war ein französischer Mathematiker.}.

\section{Definition - hermitesch}
Eine \(n \times n\)-Matrix \(A\) mit Einträgen in \(\mathrm{C}\) heißt hermitesch, wenn sie mit
ihrer (hermitesch) Adjungierten \(A^*\), also der transponierten und komplex konjugierten Matrix
übereinstimmt. Das heißt, wenn
\begin{equation}
    A = A^* = \overline A^{\mathrm T} = \overline{A^{\mathrm T}}
\end{equation}
gilt.

Für die adjungierte Matrix finden sich auch die Bezeichnungen \(A^\dagger\) und \(A^{\mathrm H}\).
Die Schreibweise \(A^*\) wird auch für die komplex konjugierte Matrix verwendet.

Für die Einträge einer hermiteschen Matrix gilt also:
\begin{equation}
\label{eq:elements}
    a_{jk} = \overline{a_{kj}}
\end{equation}
Anders formuliert ist eine Matrix A genau dann hermitesch, wenn ihre Transponierte gleich ihrer
komplex Konjugierten ist, d.h. \(A^{\mathrm T} = \overline A\).

\subsection{direkte Folgen aus der Definition}
\begin{itemize}
  \label{item:hermSym}
  \item Der Realteil einer hermiteschen Matrix ist symmetrisch, \(\mathrm{Re}(a_{jk}) =
        \mathrm{Re}(a_{kj})\), der Imaginärteil ist schiefsymmetrisch, \(\mathrm{Im}(a_{jk}) =
        -\mathrm{Im}(a_{kj})\).
  \item Ist \(A\) hermitesch, dann ist \(iA\) schiefhermitesch.
  \item Die Hauptdiagonalelemente einer hermiteschen Matrix sind reell.  
  \item Für reelle Matrizen fallen die Begriffe hermitesch und symmetrisch zusammen.
  \item Hermitesche Matrizen sind normal, d. h \(A^* \cdot A = A\cdot A^*\)
\end{itemize}

\section{Mathematischer Beweis für die gegebene Matrize 'H'}
Mit der gegebenen Bildungsvorschrifft
\begin{equation}
    H := \left(\frac{1}{j+k+1}\right)_{j,k=0}^{n}
\end{equation}
Gleichung \ref{eq:elements} und den Aussagen aus \ref{item:hermSym} folgt, dass für \(H\) der
folgende Zusammenhang gillt:
\begin{align}
                                       H &= H^{\mathrm T} &&| \text{ da rein reelwertig}      \\
                                 H_{j,k} &= H_{k,j} &&| \text{ Indizees vertauschbar}         \\
\left(\frac{1}{j+k+1}\right)_{j,k=0}^{n} &= \left(\frac{1}{j+k+1}\right)_{k,j=0}^{n} &&
\end{align}
somit ist \(H\) symmetrisch und auch hermitesch.
	\chapter{Eigenwerte und Kondition von H}
Die Erstellung der Matrix \(H\) wird durch ein Octave-Script\footnote{GNU Octave ist eine freie
Software zur numerischen Lösung mathematischer Probleme, wie zum Beispiel Matrizenrechnung, Lösen
von (Differential-)Gleichungssystemen, Integration etc. Berechnungen können in Octave mit einer
Skriptsprache durchgeführt werden, die weitgehend zu dem proprietären MATLAB kompatibel ist.}
organisiert. Siehe hierzu \ref{code:foo} und \ref{code:numa}.

Die berechneten Daten werden in zwei Dateien geschrieben,
\lstset{language=Octave, inputencoding=utf8, literate={Ö}{{\"O}}1 {Ä}{{\"A}}1 {Ü}{{\"U}}1
{ß}{{\ss}}2 {ü}{{\"u}}1 {ä}{{\"a}}1 {ö}{{\"o}}1}
\begin{lstlisting}
    # öffnen und schreiben in eine Datei
    # mögliche Modi: r-read, w-write, a-append
    fileId = fopen ('fileName','mode');
    # do something
        ...
    # done!
    # schließen der Datei nicht vergessen
    fclose (fileId);
\end{lstlisting}
um sie später mit einem externen Programm (z.B. gnuplot) weiter verarbeiten zu können. Ihnen sind
die Werte in Tabelle \ref{tab:task3} entnommen.

\vspace{1em}
\centering    
\begin{table}[htbp]
\tiny
\caption{Eigenwerte \& Konditionen für 3..10xn Matrizen \label{tab:task3}}         
\begin{tabularx}{\textwidth}{|c|X|X|X|X|X|X|X|X|X|X|X|}
    \hline    
    N &&&&&&&&&&& Kond.\\\hline
    3 & 0.00268 & 0.12232 & 1.40831 &&&&&&&& 524.06\\\hline
    4 & 9.6702 e-05 & 6.7383 e-03 & 1.6914 e-01 & 1.5002 e+00 &&&&&&& 1.5514 e+04\\\hline
    5 & 3.2879 e-06 & 3.0590 e-04 & 1.1407 e-02 & 2.0853 e-01 & 1.5671 e+00 &&&&&& 4.7661 e+05\\\hline
    6 & 1.0828 e-07 & 1.2571 e-05 & 6.1575 e-04 & 1.6322 e-02 & 2.4236 e-01 & 1.6189 e+00 &&&&& 1.4951 e+07\\\hline
    7 & 3.4939 e-09 & 4.8568 e-07 & 2.9386 e-05 & 1.0086 e-03 & 2.1290 e-02 & 2.7192 e-01 & 1.6609 e+00 &&&& 4.7537 e+08\\\hline
    8 & 1.1115 e-10 & 1.7989 e-08 & 1.2943 e-06 & 5.4369 e-05 & 1.4677 e-03 & 2.6213 e-02 & 2.9813 e-01 & 1.6959 e+00 &&& 1.5258 e+10\\\hline
    9 & 3.4998 e-12 & 6.4609 e-10 & 5.3856 e-08 & 2.6730 e-06 & 8.7581 e-05 & 1.9789 e-03 & 3.1039 e-02 & 3.2163 e-01 & 1.7259 e+00 && 4.9315 e+11\\\hline
    10& 1.0931 e-13 & 2.2667 e-11 & 2.1474 e-09 & 1.2290 e-07 & 4.7297 e-06 & 1.2875 e-04 & 2.5309 e-03 & 3.5742 e-02 & 3.4293 e-01 & 1.7519 e+00 & 1.6025 e+13\\\hline
\end{tabularx}
\end{table}
	\chapter{Numerische Lösung des LGS für eine fünfstellige Genauigkeit von R(0)}
Falls H eine \(n \times n\)-Matrix ist, wird von Octave Gaußelimination verwendet, sonst wird via
QR-Zerlegung eine Lösung im Sinne der kleinsten Fehlerquadrate berechnet. Ist A schlecht
konditioniert oder singulär, wird eine Warnung ausgegeben. Um zu zeigen, dass das
Eliminationsverfahren nach Gauß ungeeignet ist haben wir hierführ eine eigene Funktion gechrieben,
welche im Listing \ref{code:gauss} zu sehen ist.

\lstset{label=code:gauss, caption=Gauß-Eliminationsverfahren als Matlab-/Octavescript}
\lstinputlisting{../src/gaussElim.m}
	\chapter{Visualisierung der Koeffezienten sowie des Taylorpolynoms}
Das resultierende Polynom \(p\left(x\right)\) welches mittels der Gauß'elemination gefunden wurde
ist in Abbildung \ref{fig:TaylorVsGauss} dargestellt. Wie zu sehen ist, oszilliert diese Lösung sehr
stark und ist somit unbrauchbar.

\begin{figure}[H]
    \vspace{-1em}
    \begin{center}
        \includegraphics[width=\textwidth]{img/aufgabe5.pdf}
    \end{center}
    \vspace{-1em}
    \caption{Plot des Taylorpolynoms vs. Gauss-Elimination}
    \label{fig:TaylorVsGauss}
\end{figure}

\newpage
\section{Ausblick}
Um das schlecht konditionierte Gleichungssystem dennoch lösen zu können, bietet sich das
Regularisierungsverfahren nach Tikhonov\footnote{Andrei Nikolajewitsch Tichonow (russisch, wiss.
Transliteration Andrej Nikolaevič Tichonov; englische
Transliteration Andrei Tikhonov; *30. Oktober 1906 in Gschatsk; †7. November 1993 in Moskau) war ein
russischer Mathematiker. Oft wird auch die Schreibung Tychonoff verwendet} an. An dieser stelle soll
jedoch dieses nicht weiter erläutert, sondern nur dessen Implementierung in Octave aufgezeigt
werden. Für das Listing siehe \ref{code:tikhonov} das hiermit resultierende Polynom ist in Abbildung
\ref{fig:Tikonov} zu sehen.

\begin{figure}[H]
    \vspace{-1em}
    \begin{center}
        \includegraphics[width=\textwidth]{img/tikhonov.pdf}
    \end{center}
    \vspace{-1em}
    \caption{Plot der Tikhonov Regularisierung}
    \label{fig:Tikonov}
\end{figure}
	%% ...

	%% Beginn vom Anhang
	\appendix
	%\pagenumbering{romanian}
	%\clearpage\setcounter{page}{2}

	%% Glossar
	%\include{glossar}

	%% Verzeichnisse
	\newpage
	\listoffigures
 	% \listoftables
	\lstlistoflistings

	%% Literaturverzeichnis
	%\newpage
	%\bibliographystyle{gerapali}
	%\bibliography{quellen}
	%\nocite{*}

	%% sonstiger Anhang:
	\chapter{Anhang}
%-------------------------------------------------------------------------------------------------%
% - Quelltexte - %
%-------------------------------------------------------------------------------------------------%
\section{Quelltexte}
\lstset{language=Octave, inputencoding=utf8, literate={Ö}{{\"O}}1 {Ä}{{\"A}}1 {Ü}{{\"U}}1
{ß}{{\ss}}2 {ü}{{\"u}}1 {ä}{{\"a}}1 {ö}{{\"o}}1}

\lstset{label=code:foo, caption=Foo}
\lstinputlisting{../src/foo.m}

\lstset{label=code:numa, caption=Berechnungen für eine nxn Matrize}
\lstinputlisting{../src/numa.m}

% \lstset{label=code:headerglobal, caption=Globaler Header}
% \lstinputlisting{code/global.h}

\end{document}