\chapter{Entwickle die Funktion \emph{f} aus obigem Problem in eine Taylorreihe}

\section{Die Taylorreihe}
In der Analysis verwendet man Taylorreihen (auch Taylor-Entwicklungen, nach dem Mathematiker Brook
Taylor\footnote{Brook Taylor (* 18. August 1685 in Edmonton, Middlesex; † 29. Dezember 1731 in
Somerset House, London) war ein britischer Mathematiker. Nach ihm wurde u.a. die Taylorreihe
benannt.}), um Funktionen in der Umgebung bestimmter Punkte durch Potenzreihen darzustellen
(Reihenentwicklung). So kann ein komplizierter analytischer Ausdruck durch eine nach wenigen
Gliedern abgebrochene Taylorreihe (oftmals gut) angenähert werden, z.B. in der Physik oder bei der
Ausgleichung geodätischer Netze: So ist die oft verwendete Kleinwinkelnäherung des Sinus eine nach
dem ersten Glied abgebrochene Taylorreihe dieser Funktion.

\section{Definition}
Sei \( I \subset \mathrm{R} \) ein offenes Intervall, \( f \colon I \rightarrow \mathrm{R} \) eine
unendlich oft differenzierbare Funktion und \(a\) ein Element von \(I\). Dann heißt die unendliche
Reihe
\begin{align}
        \label{eq:taylorreihe1}
P_f(x)  & = f(a) + \frac{f'(a)}{1!} (x-a) + \frac{f''(a)}{2!} (x-a)^2 + \dotsb +
            \frac{f^{(n)}(a)}{n!} (x-a)^n + \dotsb\\
        \label{eq:taylorreihe2}
        & = \sum_{n=0}^\infty \frac{f^{(n)}(a)}{n!} (x-a)^n
\end{align}
die Taylor-Reihe von f mit Entwicklungsstelle a.

Hierbei bezeichnet
\begin{itemize}
    \item   \(n!\) die Fakultät \(n! = 1 \cdot 2 \cdot \dots \cdot n\) und
    \item   \(f^{(n)}(a)\) die n-te Ableitung von \(f\) an der Stelle a, wobei man \(f^{(0)} := f\)
            setzt.
\end{itemize}

\section{Eigenschaften}
Die Taylorreihe \(P_f(x)\) zur Funktion \(f(x)\) ist eine Potenzreihe \(P\) in \(x\). Im Fall einer
analytischen Funktion \(f\) hat die Taylor-Reihe in jedem Punkt \(a\) einen positiven
Konvergenzradius \(r\) und stimmt in ihrem Konvergenzbereich mit \(f\) überein, d.h. es gibt ein
\(r>0\), sodass für \(|x-a|<r\) gilt
\begin{equation}
    P_f(x) = \sum_{n=0}^\infty a_n (x-a)^n = f(x)
\end{equation}.
Außerdem stimmen die Ableitungen der Reihe im Entwicklungspunkt \(a\) mit den tatsächlichen
Ableitungswerten überein:
\begin{equation}
    P_f^{(k)}(a) = f^{(k)}(a)
\end{equation}
.
    
\section{Konstruktion}
Damit die Ableitungen der Reihe im Entwicklungspunkt \(a\) mit den tatsächlichen Ableitungswerten
übereinstimmen, sind die Koeffizienten \(a_n\) der Taylorreihe wie folgt konstruiert:
\begin{equation}
    \label{eq:konstruktion}
    a_n = \frac{f^{(n)}(a)}{n!} \Rightarrow P_f(x) = \sum_{n=0}^\infty \frac{f^{(n)}(a)}{n!} (x-a)^n
\end{equation}
.

\section{Entwicklung von \emph{f(x)} in eine Tylor Reihe}
Bei gegebener Funktion
\begin{equation}
    f\left(x\right) = \frac{1}{1+x}
\end{equation}
und den Gleichungen \ref{eq:taylorreihe1} und \ref{eq:taylorreihe2} folgt für deren Ableitungen
\begin{align}
    f'\left(x\right)    &= -\frac{1}{\left(1+x\right)^2} \\
    f''\left(x\right)   &=  \frac{2\left(1+x\right)}{\left(1+x\right)^4}
                         =  \frac{2}{\left(1+x\right)^3} \\
    f'''\left(x\right)  &=  \frac{2\left(1+x\right)^4 -8\left(1+x\right)\left(1+x\right)^3}
                            {\left(1+x\right)^8}
                         =  -\frac{6}{\left(1+x\right)^4}
\end{align}
Hieraus läßt sich folgende Bildungsvorschrifft für eine beliebige Ableitung bestimmen
\begin{equation}
    f^n\left(x\right) = \left(-1\right)^n \frac{n!}{\left(1+x\right)^{n+1}}
\end{equation}
hieraus folgt nach Gleichung \ref{eq:konstruktion}
\begin{align}
    f\left(x\right)  =  P_f\left(x\right) 
                    &= \sum_{n=0}^{\infty} \left(-1\right)^n
                        \frac{n!}{n!\left(1+a\right)^{n+1}}\left(x-a\right)^n\\
                    &=  \sum_{n=0}^{\infty} \left(-1\right)^n
                        \frac{\left(x-a\right)^n}{\left(1+a\right)^{n+1}}
\end{align}